\documentclass[11pt]{article}

% Preamble!!
\usepackage{titling}
\setlength{\voffset}{0.7in}
\setlength{\droptitle}{-10em}
\usepackage{titlesec}
\titlelabel{\thetitle.\quad}
\usepackage{xcolor}
\usepackage{adjustbox}
\usepackage{amsmath, amsthm, amsfonts, amssymb}
\usepackage{graphicx}
\usepackage{setspace}
\usepackage{longtable}
\usepackage{breqn}
\usepackage{lscape}
\usepackage{indentfirst}
\usepackage[labelsep=period,justification=justified,singlelinecheck=false,font = footnotesize, labelfont=bf]{caption}
\usepackage{booktabs}
\usepackage{tabularx,ragged2e}
\usepackage{natbib}
\usepackage{rotating}
\usepackage{placeins}
\usepackage{subcaption}
\usepackage{hyperref}
\definecolor{burntorange}{rgb}{0.8, 0.33, 0.0}
\hypersetup{
    colorlinks=true,
    linkcolor=orange,
    filecolor=magenta,      
    urlcolor=burntorange,
}
\pagenumbering{arabic}

\usepackage{bbm}
\usepackage[margin=1in]{geometry}

\usepackage{enumerate}
\usepackage{array}
\usepackage[T1]{fontenc}
\usepackage[font=small,labelfont=bf,tableposition=top]{caption}
\usepackage{mathtools}
\newcommand\eho{\stackrel{\mathclap{\small\mbox{$H_0$}}}{=}}
\newcommand\sho{\stackrel{\mathclap{\small\mbox{$*$}}}{=}}
\newcommand\dho{\stackrel{\mathclap{\small\mbox{$d$}}}{=}}
\newcommand\qho{\stackrel{\mathclap{\small\mbox{$?$}}}{=}}

% \usepackage{authblk}


\DeclareCaptionLabelFormat{andtable}{#1~#2  \&  \tablename~\thetable}

\usepackage{fullpage, amsmath, amssymb, amsthm, bbm, color}
\usepackage{graphicx,caption,subcaption,placeins}
\usepackage{dsfont}

\usepackage{natbib}
\bibpunct{(}{)}{;}{a}{,}{,}

\newtheorem{assumption}{Assumption}
\newtheorem{definition}{Definition}
\newtheorem{theorem}{Theorem}
\newtheorem{example}{Example}
\newtheorem{procedure}{Procedure}


\usepackage{tikz}
\usetikzlibrary{positioning,chains,fit,shapes,calc}

\definecolor{myblue}{RGB}{80,80,160}
\definecolor{mygreen}{RGB}{80,160,80}


\begin{document}

\title{FIN 373 Homework 8 \\ {\large due: \textbf{10/26/21}}}
\date{}
\maketitle

\vspace{-20mm}

\noindent Instructions: Please submit solutions on canvas.  Only a knitted pdf of an {\tt Rmarkdown} file will be accepted.
\\

\noindent \textbf{Problem 1:} In this exercise we're going to look at the effect 
of a educational television program 
\href{https://en.wikipedia.org/wiki/The_Electric_Company}{The Electric Company}
that ran from 1971-1977 on children's reading scores.  We will investigate what reading gains, if any, were made by the 1st through 4th grade classes as part of a randomized experiment.\footnote{This exercise is based on \href{https://files.eric.ed.gov/fulltext/ED130635.pdf}{The Electric Company: Television and Reading, 1971-1980: A Mid-Experiment Appraisal}. Joan G. Cooney (1976). Children's Television Network Report.}

The data comes from a two location trial in which 
treatment was randomized at the level of school classes.\footnote{Classes were 
paired, but we will ignore that in the analysis.} Each class was 
either treated (to watch the program) or control (to not watch the 
program). The outcome of interest is the score on a reading test 
administered at the end of each year called {\tt post.score}. Note that 
these are distinct classes from all four years.  The variables in {\tt electric-company.csv} are:



\vspace{3mm}
\begin{center}
\begin{tabular}{l p{13. cm}}
 \hline
\textit{Variable} & \textit{Description} \\
\hline
{\tt pair} &               The index of the treated and control pair (ignored
                      here). \\
{\tt city} &               The city: Fresno ("F") or Youngstown ("Y") \\
{\tt grade} &              Grade (1 through 4) \\
{\tt supp} &               Whether the program replaced (``R'') or supplemented 
                      (``S'') a reading activity \\
{\tt treatment} &          ``T'' if the class was treated, ``C'' otherwise (randomized) \\
{\tt pre.score} &          Class reading score \textit{before} treatment, at the 
                      beginning of the school year \\
{\tt post.score} &        Class reading score at the end of the school year \\
\hline
\end{tabular}
\end{center}    

\vspace{1mm}  
\begin{enumerate}[a.]
	\item Read the data into an data frame named {\tt electric}. 
What sort of variable has R assumed {\tt grade} is? How will 
it be treated in a linear model if we use it as an independent variable? 
Under what circumstances would that be reasonable or unreasonable?

Make a new variable from {\tt grade} that is a factor. How will a 
linear model treat this new variable? (\textit{Hint}: You may 
find that {\tt summary()} illuminates the new data set).

Finally, overwrite the existing {\tt treatment} variable so that it is 
numerical: 1 when the class is treated and 0 when not.

\item Let's now consider the effect of {\tt treatment}. First, fit a linear model 
that predicts {\tt post.score} with just the {\tt treatment}. Then fit a model uses 
your factor version of {\tt grade} as well as {\tt treatment}.  

Summarize both models in terms of how much of the variance 
in {\tt post.score} they ``explain''. 

Now, consider each model's {\tt treatment} coefficient. 
Are the estimates of this coefficient 
\textit{different} in the two models? Why do you think that is?

\item Now make another model that uses the factor version of {\tt grade} and 
{\tt pre.score} (the reading score before the year begins) to predict 
{\tt post.score}.  Is this model better? If so, in what ways?

\item Now let's consider the effect of treatment \textit{within} each grade. 
We can use the {\tt lm} function's ``subset'' argument to fit the model on just a subset of all the rows in the data set. For example, we can fit a model 
of the relationship of {\tt post.score} to {\tt treatment} and {\tt pre.score} just in grade 2 like this:

\vspace{2mm}
{\tt 
mod <- lm(post.score ~ treatment + pre.score, data = electric, 
          subset = grade == 2)
}

\vspace{-2mm}
Fit a linear model predicting {\tt post.score} using {\tt treatment} and 
{\tt pre.score} for each 
grade. Follow the procedure below:


\begin{enumerate}
\item Define a function named {\tt fit_reg()} that returns the coefficient on {\tt treatment}. The function should have two arguments: the entire data ({\tt data_all}) and the grade ({\tt grade_subset}).
\item Use a for loop and call the {\tt fit_reg()} function for each grade (1 to 4).
\item Print out the coefficient on treatment using the {\tt print()} function.
   Store what the {\tt fit_reg()} function returns in a variable.
\item Briefly comment on the result. There are now \textit{four} treatment effects. How do they differ as grade increases?
\end{enumerate}

\item Finally, let's investigate the separate grade effects in a single model. One 
way to do this is to \textit{interact} {\tt treatment} with {\tt grade}. Here's a general 
modeling principle: If you think the \textit{effect} of variable {\tt A} varies according to the \textit{values} of variable {\tt B}, then you should consider adding an interaction between {\tt A} and  {\tt B} in your model.  In the {\tt lm()} function this amounts to adding an {\tt A:B} term. For example,
if {\tt A} and {\tt B} interact to predict Y, then the formula would be, 
{\tt
Y $\sim$ A + B + A:B},
which would fit the model 
$$
Y_i = \beta_0 + \beta_A A_i + \beta_B B_i + \beta_{AB} (A_i \times B_i)  + \epsilon_i.
$$
An alternative syntax to fit this model is to use the right-hand-side expression {\tt A*B}. So, to fit the model above using this notation the formula is 
{\tt 
Y $\sim$ A * B. 
}
This will automatically include the {\tt A}, {\tt B}, and {\tt A:B} terms in the model!

Fit a model of all the grades that includes {\tt pre.score}, {\tt treatment}, {\tt grade} (factor version), the factor version of {\tt grade} interacted with {\tt treatment}, and the factor version of {\tt grade} interacted with {\tt pre.score}
(this is called a fully interacted model). 
How would you construct grade-specific treatment effects from these coefficients?
Show an example for grade 2.

\end{enumerate}









\end{document}

